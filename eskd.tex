\secru{Перечень стандартов ЕСКД}\label{eskd}

\href{http://rtu.samgtu.ru/sites/rtu.samgtu.ru/files/GOST_ESKD_2.701-2008.pdf}{ГОСТ
2.701-2008} Схемы. Виды и типы. Общие требования к выполнению.

\href{http://www.gostedu.ru/51102.html}{ГОСТ 2.702-2011} ЕСКД
Правила выполнения электрических схем.

ГОСТ 2.703-68  Правила выполнения кинематических схем.

ГОСТ 2.704-76  Правила выполнения гидравлических и кинематических схем.

ГОСТ 2.705-70  Правила выполнения электрических схем обмоток и изделий с обмотками.

ГОСТ 2.708-81  Правила выполнения электрических схем цифровой вычислительной техники.

ГОСТ 2.709-72 Система обозначения цепей в электрических схемах.

\href{http://www.electromonter.info/handbook/symbol\_271081.html}{ГОСТ 2.710-81}
Обозначения буквенно-цифровые в электрических схемах.

ГОСТ 2.711-82 Схема деления изделия на составные части.

ГОСТ 2.721-74 Обозначения общего применения.

ГОСТ 2.722-68 Машины электрические.

ГОСТ 2.723-68 Катушки индуктивности, дроссели, трансформаторы, автотрансформаторы и магнитные усилители.

ГОСТ 2.725-68 Устройства коммутирующие.

ГОСТ 2.726-68 Токосъемники.

ГОСТ 2.727-68 Разрядники, предохранители.

\href{http://www.bmstu.ru/~rl1/courses/inform/gost2\_728-74.pdf}{ГОСТ 2.728-74}
Резисторы, конденсаторы.

ГОСТ 2.729-68 Приборы электроизмерительные.

ГОСТ 2.730-73 Приборы полупроводниковые.

ГОСТ 2.731-81 Приборы электровакуумные.

ГОСТ 2.732-68 Источники света.

ГОСТ 2.733-68 Обозначения условные графические детекторов ионизирующих излучений в схемах.

ГОСТ 2.734-68 Линии сверхвысокой частоты и их элементы.

ГОСТ 2.735-68 Антенны.

ГОСТ 2.736-68 Элементы пьезоэлектрические и магнитострикционные. Линии задержки.

ГОСТ 2.737-68 Устройства связи.

ГОСТ 2.741-68 Приборы акустические.

ГОСТ 2.742-68 Источники тока электрохимические.

\href{http://www.bmstu.ru/~rl1/courses/inform/gost2\_743-91.pdf}{ГОСТ
2.743-91} Элементы цифровой техники.

ГОСТ 2.744-68 Устройства электрозапальные.

ГОСТ 2.745-68 Электронагреватели, устройства и установки электротермические.

ГОСТ 2.746-68 Генераторы и усилители квантовые.

ГОСТ 2.747-68 Размеры условных графических обозначений.

ГОСТ 2.751-73 Обозначение линий, жгутов и кабелей.

ГОСТ 2.755-87 Устройства коммутационные и контактные соединения.

ГОСТ 2.756-76 Воспринимающая часть электромеханических устройств.

ГОСТ 2.758-81 Сигнальная техника.

ГОСТ 2.759-82 Элементы аналоговой техники.

ГОСТ 2.761-84 Компоненты световодных систем.

ГОСТ 2.762-85 Частоты и диапазоны частот с частотным разделением каналов.


