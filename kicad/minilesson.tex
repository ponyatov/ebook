\secru{Создание проекта в менеджере проектов \file{kicad}}

Лучше всего для каждого проекта использовать раздельные папки; в противном
случае система может сбиться с толку, если файлы из разных проектов будут лежать
в одной папке. Проделайте следующие шаги:

\begin{enumerate}
  \item Создайте папку проекта \file{D:/ARM/SpindleDriver}
  \item Запустите программу KiCad
  \item Создайте проект (project)
  \begin{itemize}
    \item 
На панели инструментов KiCad выберите левую иконку с подсказкой\\
\menu{Начать новый проект}, используйте команду меню
\menu{Файл>Новый>Пустой} или сочетание клавиш \keys{Ctrl+N}.
    \item 
В диалоге \menu{Создать новый проект} выберите созданную папку
выберите только что созданную папку \file{D:/ARM/SpindleDriver} и
введите имя проекта \menu{\file{SpindleDriver}} и нажмите \menu{Сохранить}.
	\item
Если папка проекта содержит какие-то файлы, будет выведено окно выбора:
создать подпапку с именем проекта \menu{Yes}, или записать файл проекта
в указанную папку как есть \menu{No}. Нажмите No.
    \item 
Сохраните проект кнопкой \menu{Сохранить текущий проект}, \menu{Файл>Сохранить}
или \keys{Ctrl+S}.
	\item
В папке появится файл \file{SpindleDriver.pro}, содержащий установки вашего 
проекта. Файл имеет тектовый формат, поэтому при необходимости его можно открыть
в любом редакторе и вручную аккуратно подправить, например скорректировать
настройки зазоров печатной платы.
  \end{itemize}
\end{enumerate}

В правой части панели имеются четыре большие кнопки запуска компонентов KiCad.
Слева направо, это:

\begin{itemize}
\item EeSchema\ --- Редактор принципиальных схем
\item CvPcb\ --- Программа редактирования падстеков (отверстий и площадок)
\item Pcbnew\ --- Редактор печатных плат
\item GerbView\ --- Программа просмотра фотошаблонов в формате Gerber
\item Bitmap2Component\ --- Создание компонента из черно-белого изображения
(например логотипа)
\item PcbCalculator\ --- Калькулятор для печатных плат
\end{itemize}

Каждая кнопка запускает соответствующую программу. Мы будем использовать эти
программы по мере изучения.

\secru{Создание принципиальной схемы в \file{eeschema}\ (часть 1)}

Запустите редактор принципиальных схем, нажав на панели KiCad большую кнопку
\menu{EeSchema}.

При первом запуске EeSchema стартует с новым проектом и показывает
предупреждение, что файла схемы еще нет. Просто нажмите \menu{ОК}.

Если вас не устраивает черный фон рабочец области или цвета элементов схемы,
поменяйте настроки цветов \menu{Настройки>Цвета}. 

На правом краю окна редактора схем есть вертикальная панель инструментов,
которые мы и будем использовать для рисования схемы. Этими инструментами можно
выбирать объекты, размещать компоненты, вводить связи и т.д.

\includegraphics[width=0.9\textwidth]{kicad/ee15.png}

Завершение работы инструмента: вы можете выбрать другой инструмент из правой
инструментальной панели или же указать \menu{Отложить инструмент} по правому
клику мышки \keys{\rms}.

\secdown

\secru{Инструмент \emph{Добавить компоненты}}

\begin{itemize}
  \item 
На правой панели нажмите кнопку \menu{Разместить компонент}\
\includegraphics[height=2em]{kicad/ee21.png}. Курсор изменится со стрелки на
карандаш. Удобнее использовать сочетание клавиш \keys{Shift+A}.
Кликните в поле схемы чтобы начать размещение компонента. Появится диалог
\menu{Выбор компонента}. Вы можете выбрать компонент несколькими путями:
  \item
  \begin{enumerate}
    \item 
Если вы знаете точное имя копонента, введите его в поле \menu{Имя}, а
затем нажмите \keys{Enter} или \keys{OK}.
    \item 
Если вы знаете имя только приблизительно, в поле \menu{Имя} введите шаблон для
поиска, например, \menu{*BD*}, затем нажмите \keys{Enter} или \keys{OK}. Вы
увидите окно \\\menu{Выбрать компонент} со списком найденных компонентов.

\includegraphics[height=0.5\textheight]{kicad/ee16.png}
    \item 
Вы можете искать компонент по ключевому слову, введя его в поле \menu{Имя},
затем кликнув \menu{Поиск по ключевому слову}. Однако на данный момент качество
библиотек все еще низкое, и немногие компоненты имеют ключевые слова, поэтому
эта возможность полезна косвенно.
    \item 
Можно выбрать недавно использованные компоненты из \menu{Списка истории}.
    \item 
Кнопка \menu{Список всех} вызывает диалог, в котором можно выбрать сначала
библиотеку \menu{74xx}, а затем ее компонент \menu{74HCT04}.
    \item 
Кнопка \menu{Выбор просмотром} вызывает \menu{Обзор библиотек}, позволяя
просмотреть библиотеки и находящиеся в них условные графические изображения.

\includegraphics[height=0.5\textheight]{kicad/ee19.png}

  \end{enumerate} 
\end{itemize}

Вы также можете
вызвать обозреватель библиотек кнопкой\\
\menu{Просмотр библиотек и
компонентов}\ \includegraphics[height=2em]{kicad/ee20.png}

Выбрав элемент \dblms, вставьте символ в нужное место схемы \lms.
Позже вы сможете переместить его если нужно.
Зеркальное отражение компонента можно произвести следующим образом:

\begin{itemize}
  \item Поместите курсор на компоненте.
  \item По \rms\ выберите \menu{Ориентация компонента>Отражение}. 
  \item Без использования \term{контекстного меню}\ --- наведите мышь на
  компонент и нажмите кнопку \keys{X}\ или \keys{Y}.
\end{itemize}

\secup

\secru{Библиотеки компонентов}

В составе KiCad поставляются библиотеки электронных компонентов 
(обычных и поверхностно монтируемых SMD). Для многих библиотечных 
компонентов есть 3D-модели, созданные в Wings3D.

Компоненты и посадочные места корпусов можно ассоциировать с документацией, 
ключевыми словами и осуществлять быстрый поиск компонента по 
функциональному назначению.


