\secru{Аппаратная платформа}

Далее нам необходимо указать, на каком конкретном железе мы хотим запускать наше
приложение. При этом важно не только какой процессор на нем стоит, но также
нужно правильно сконфигурировать ядро \linux, чтобы были включены \emph{все
необходимые драйвера}, также в целевую сборку должен быть подключен правильный
скрипт инициализации оборудования и добавлены платформенно-специфичные
программы.

\lstinputlisting[style=mk,title=Makefile.mk,
firstnumber=6,linerange={6-10}]{cross/Makefile.mk}

С текущей версией книги в каталоге \file{cross}\ в комплекте идут наборы
скриптов сборки, поддерживающие следующие HW-конфигурации:

\begin{itemize}
  \item \file{x86pc} Типовой ПК на процессоре архитектуры i386, конфигурация
  прописана максимально совместимо, поэтому в результате сборка запустится на
  любом wintel компьютере, но не будут работать многие драйвера, потому что их
  нет. По необходимости вам нужно самостоятельно создать и прописать новую
  HW-конфигурацию для вашего железа, если захочется запустить звук, сеть и т.п.
\lstinputlisting[style=mk,title=hw/x86pc.mk]{cross/hw/x86pc.mk}
\lstinputlisting[style=mk,title=cpu/i386/generic486dx.mk]{cross/cpu/i386/generic486dx.mk}
  \item \file{eeepc701}
  Полная конфигурация для нетбука ASUS EeePC 701 4G
\lstinputlisting[style=mk,title=hw/eeepc701.mk]{cross/hw/eeepc701.mk}
\lstinputlisting[style=mk,title=cpu/i386/celeronMulv353.mk]{cross/cpu/i386/celeronMulv353.mk}
  \item \file{qemu386}
  Конфигурация для запуска системы на эмуляторе QEMU в режиме i386, включены
  все необходимые драйвера, совместимые с эмулятором.
\lstinputlisting[style=mk,title=hw/qemu386.mk]{cross/hw/qemu386.mk}
  \item \file{coubie1}
  Конфигурация для платы
  \href{http://linux-sunxi.org/Cubietech\_Cubieboard}{CoubieBoard} версии 1,
  версии 2 у меня нет для тестирования.
  Подробно опции компиляции описаны \href{http://habrahabr.ru/post/146877/}{тут}
\lstinputlisting[style=mk,title=hw/coubie1.mk]{cross/hw/coubie1.mk}
\lstinputlisting[style=mk,title=cpu/arm/sun4i.mk]{cross/cpu/arm/sun4i.mk}
  \item \file{rpi}
  Конфигурация для платы \raspi
\lstinputlisting[style=mk,title=hw/rpi.mk]{cross/hw/rpi.mk}
\lstinputlisting[style=mk,title=cpu/arm/BCM2835.mk]{cross/cpu/arm/BCM2835.mk}
\cp{http://www.gurucoding.com/en/rpi\_cross\_compiler/building\_binutils\_gcc1\_cygwin\_vm.php}

В случае необходимости откатиться на эмулятор плавающей дочки добавляется
\lstinputlisting[style=mk,title=cpu/arm/softfp.mk]{cross/cpu/arm/softfp.mk}

\end{itemize}

Связка из переменной \verb|HW|, и файлов \verb|hw.$(HW).mk|, 
\verb|cpu.$(CPU).mk| и \verb|arch.$(ARCH).mk| задают значения набора глобальных
переменных, указывающих полную спецификацию на платформу:

\begin{enumerate}
  \item \verb|HW| краткое название целевой железки
  \item \verb|CPU| целевой процессор
  \item \verb|CFG_CPU| параметры сборки тулчейна для процессора
  \item \verb|CFLAGS_CPU| параметры компилятора для настройки генерации кода 
  \item \verb|ARCH| архитектура процессора
\lstinputlisting[style=mk,title=arch/i386.mk]{cross/arch/i386.mk}
\lstinputlisting[style=mk,title=arch/arm.mk]{cross/arch/arm.mk}
  \item \verb|TARGET| \term{триплет} целевой платформы в формате GNU
\end{enumerate}
