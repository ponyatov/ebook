\secru{Аппаратная платформа}

Далее нам необходимо указать, на каком конкретном железе мы хотим запускать наше
приложение. При этом важно не только какой процессор на нем стоит, но важно
правильно сконфигурировать ядро \linux, чтобы были включены \emph{все
необходимые драйвера}, также в целевую сборку должен быть подключен правильный
скрипт инициализации оборудования и добавлены платформенно-специфичные
программы.

\lstinputlisting[style=mk]{cross/hw.mk}

С текущей версией книги в каталоге \file{cross}\ в комплекте идут наборы
скриптов сборки, поддерживающие следующие архитектуры:

\begin{itemize}
  \item \file{x86pc} Типовой ПК на интеловском процессоре, конфигурация
  прописана максимально совместимо, поэтому в результате сборка запустится на
  любом wintel компьютере, но не будут работать многие драйвера, потому что их
  нет. По необходимости вам нужно самостоятельно создать и прописать новую
  HW-конфигурацию для вашего железа, если захочется запустить звук, сеть и т.п.
  \item \file{qemu386}
  Конфигурация для запуска системы на эмуляторе QEMU в режиме i386, включены
  все необходимые драйвера, совместимые с эмулятором.
  \item \file{coubie1}
  Конфигурация для платы CoubieBoard версии 1, версии 2 у меня нет для
  тестирования.
  \item \file{rpi}
  Конфигурация для платы \raspi
\end{itemize}
