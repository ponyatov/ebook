\secen{Embedded Linux}
\secru{Linux для встраиваемых систем}

\ru{Linux для встраиваемых систем}\ --- будем называть его \emlinux, популярный
метод быстрого создания комплекса ПО для больших сложных приложений,
работающих на достаточно мощном железе, особенно предполагающих интенсивное
использование сетевых технологий.

За счет использования уже существующей и очень большой базы исходных текстов
ядра, библиотек и программ для \linux, бесплатно доступных в т.ч. и для
коммерческих приложений, можно на порядки сократить стоимость разработки
собственных программных компонентов, и при этом получить очень мощную команду
бесплатных стронних разработчиков, уже знакомых с созданием ПО для \linux.

Из недостатков можно отметить:
\begin{itemize}
  \item Отсутствие полноценной поддержки режима жесткого реального времени;
  \item Тяжелое ядро;
  \begin{itemize}
  \item Поддерживаются только мощные семейства процессоров;
  \item Значительные требования по объему \ram\ и общей производительности;
  \end{itemize}
\ru{ \item Колхозная дремучесть техспециалистов, контуженных ТурбоПаскалем и
Win\-dows\-ом;}
\end{itemize}

Для \emph{сборки}\ \emlinux-системы используется метод чистой кросс-компиляции,
когда используется \term{кросс-тулчейн}, компилирующий весь комплект ПО для
компьютера другой архитектуры\footnote{\ типичный пример\ --- сборка ПО на ПК
c процессором Intel i7 для \raspi\ или планшета на процессоре
AllWinner/Tegra/\ldots}.

\emlinux\ очень широко применяется на рынке мобильных устройств\footnote{\ в
т.ч. является основой Android}, и устройств интенсивно использующих
сетевые протоколы (роутеры, медиацентры).
