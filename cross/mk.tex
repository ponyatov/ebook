\secrel{Makefile}

Вся сборка проекта управляется через \file{Makefile}, поделенный на несколько
частей:

\lstinputlisting[style=mk,title=Makefile.mk,
linerange={1-1}]{cross/Makefile.mk}

\secdown

\secru{Выбор приложения}

\term{Приложение}\ --- программно-аппаратный комплекс, решающий определенную
задачу. В нашем случае приложение это \linux-часы, соответственно назовем
приложение:

\lstinputlisting[style=mk,title=Makefile.mk,
firstnumber=3,linerange={3-4}]{cross/Makefile.mk}

В переменной \verb|APP|\ указывается имя приложения. 
В зависимости от приложения, в главный Makefile нужно подключить кусок
соответствующего Makefile, выполняющего какие-то действия или настройки.

\lstinputlisting[style=mk,title=Makefile.mk,
firstnumber=12,linerange={12-12}]{cross/Makefile.mk}

\lstinputlisting[style=mk,title=app.clock.mk]{cross/app.clock.mk}


\secru{Аппаратная платформа}

Далее нам необходимо указать, на каком конкретном железе мы хотим запускать наше
приложение. При этом важно не только какой процессор на нем стоит, но важно
правильно сконфигурировать ядро \linux, чтобы были включены \emph{все
необходимые драйвера}, также в целевую сборку должен быть подключен правильный
скрипт инициализации оборудования и добавлены платформенно-специфичные
программы.

\lstinputlisting[style=mk]{cross/hw.mk}

С текущей версией книги в каталоге \file{cross}\ в комплекте идут наборы
скриптов сборки, поддерживающие следующие архитектуры:

\begin{itemize}
  \item \file{x86pc} Типовой ПК на интеловском процессоре, конфигурация
  прописана максимально совместимо, поэтому в результате сборка запустится на
  любом wintel компьютере, но не будут работать многие драйвера, потому что их
  нет. По необходимости вам нужно самостоятельно создать и прописать новую
  HW-конфигурацию для вашего железа, если захочется запустить звук, сеть и т.п.
  \item \file{qemu386}
  Конфигурация для запуска системы на эмуляторе QEMU в режиме i386, включены
  все необходимые драйвера, совместимые с эмулятором.
  \item \file{coubie1}
  Конфигурация для платы CoubieBoard версии 1, версии 2 у меня нет для
  тестирования.
  \item \file{rpi}
  Конфигурация для платы \raspi
\end{itemize}

\secru{Пакеты}

В обычных дистрибутивах пакетами называют архивы скомпилированных программ,
устанавливаемых в \linux-систему. В нашем случае кросс-компиляции, будем
называть \term{пакетом}\ архив исходных текстов определенной прграммы или
компонента системы, вместе с секциеями мейкфайла, выполняющего ее компиляцию.

Также пакет может быть не связан с компиляцией ПО, а выполнять какую-то
вспомогательную работу.

Пакеты запускаются по своему имени вручную с помощью команды
\begin{lstlisting}
~/book/cross$ make -f Makefile.mk <pacname1> [<packname2>...]
\end{lstlisting}

\secdown
\secru{Версии}

Для сборки кросс-компилятора нам потребуются следующие пакеты:

\begin{itemize}
  \item \file{binutils}\ ассемблер, линкер и утилиты для работы с
  \term{объектными файлами}
  \item \file{gmp}, \file{mpfr}, \file{mpc}\ библиотеки арифметики произвольной
  точности, необходимы для компиляции \file{gcc}
  \item \file{gcc}\ набор компиляторов GNU, нам нужны только Си и C++. 
\end{itemize}

В принципе можно поставить готовые кросс-компиляторы и библиотеки из репозитория
вашего дистрибутива, и не тратить время на сборку кросс-тулчейна. Но мы это
сделаем для иллюстрации возможности собрать свой тулчейн из исходников только
для одного конкретного проекта, не интегрируя собранный тулчен глобально в вашу
\linux-систему.

Такая сборка может понадобится в некоторых специфических случаях, как то желание
использовать самые свежие версии компиляторов, или изменить опции сборки
компилятора, отличающиеся от тех, которые заданы при сборке дистрибутивного.
Например в новой версии появилась расширенная поддержка вашего процессора,
подправили оптимизацию, или вам нужно наоборот собрать несколько пакетов старых
версий, которые не желают собираться новой версией gcc.

\lstinputlisting[style=mk,title=Makefile.mk,
firstnumber=14,linerange={14-18}]{cross/Makefile.mk}

Любая \emlinux-система включает в себя следующий обязательный набор базовых
пакетов:

\begin{enumerate}
  \item \file{linux} ядро ОС \linux
  \item \file{libc} стандартная библиотека Си, будем использовать специальную
  обрезанную версию \file{uClibc}, специально заточенную под применение во
  всраиваемых системах
  \item \file{bb} пакет \file{BusyBox}, набор множества стандартных утилит,
  тоже с обрезанными функциями для экономии ресурсов 
\end{enumerate}

\lstinputlisting[style=mk,title=Makefile.mk,
firstnumber=19,linerange={19-21}]{cross/Makefile.mk}

\secru{Полные имена пакетов}

включают номер версии пакета:

\lstinputlisting[style=mk,title=Makefile.mk,
firstnumber=23,linerange={23-30}]{cross/Makefile.mk}


\secup



\secru{Директории}

в которых будет производиться сборка, и формироваться целевая файловая система.

\secdown

\secrel{PWD}

Встроенная переменная, возвращает текущий каталог, из которого быз запущен
\file{make}, т.е. \file{/home/user/book/cross}.

\secrel{GZ}

Каталог для хранения архивов исходных текстов пакетов, закачанных из \internet\
(локальное зеркало). Если вы работаете сразу с несколькими проектами, создайте
отдельный каталог, например \file{/home/user/gz}, и из каждого проекта сделайте
на него симлинк \file{ln -s /home/user/gz gz}. В этом случае все архивы окажутся
общими для всех проектов, причем там могут лежать несколько версий одного
пакета.

\lstinputlisting[style=mk,title=Makefile.mk,
firstnumber=32,linerange={32-32}]{cross/Makefile.mk}

\secrel{SRC}

Сюда будут распаковываться исходные тексты пакетов

\lstinputlisting[style=mk,title=Makefile.mk,
firstnumber=33,linerange={33-33}]{cross/Makefile.mk}

\secrel{TMP}

Здесь будут создаваться и удаляться временные файлы и каталоги при сборке
пакетов\ --- out of tree сборка, за исключением пакетов, которые могут быть
собраны только в каталоге с исходными текстами.

\lstinputlisting[style=mk,title=Makefile.mk,
firstnumber=34,linerange={34-34}]{cross/Makefile.mk}

\secrel{TC}

От ToolChain, или TargetCompiler\ --- сюда поместим собранный кросс-тулчейн,
чтобы не засорять основную \linux-систему, и иметь возможность пользоваться
разными версиями компиляторов для разных проектов

\lstinputlisting[style=mk,title=Makefile.mk,
firstnumber=35,linerange={35-35}]{cross/Makefile.mk}

\secrel{ROOT}

В этом каталоге будем собирать \term{rootfs}\ --- корневую файловую систему

\lstinputlisting[style=mk,title=Makefile.mk,
firstnumber=36,linerange={36-36}]{cross/Makefile.mk}

\secrel{BOOT}

В этот подкаталог поместим откомпилированное ядро, файлы загрузчика и другие
файлы, обеспечивающие запуск системы до момента монтирования rootfs.

\lstinputlisting[style=mk,title=Makefile.mk,
firstnumber=37,linerange={37-37}]{cross/Makefile.mk}

\secup

\secru{Макросы команд}

\lstinputlisting[style=mk,title=Makefile.mk,
firstnumber=39,linerange={39-42}]{cross/Makefile.mk}

\secru{\file{dirs}}

Создание дерева каталогов для сборки.
Каталоги не зачищаемые пакетом \file{distclean}, прописаны отдельно. 

\lstinputlisting[style=mk,title=Makefile.mk,
firstnumber=44,linerange={44-47}]{cross/Makefile.mk}

\secru{\file{distclean}}

Зачистка проекта, применяется при полной пересборке

\lstinputlisting[style=mk,title=Makefile.mk,
firstnumber=49,linerange={49-52}]{cross/Makefile.mk}

\secru{Загрузка архивов исходников\ --- \file{gz}}

Длительная и потребляющая трафик операция, нужен онлайн.
По-хорошему архив исходников тут было бы желательно загружать через пиринговые
сети, а не нагружать зеркала. Если эта книга вдруг окажется популярной, нужно
будет сделать хранение архивов на торрентах.

\lstinputlisting[style=mk,title=Makefile.mk,
firstnumber=54,linerange={54-63}]{cross/Makefile.mk}



\secup