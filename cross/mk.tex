\secrel{Makefile}

Вся сборка проекта управляется через \file{Makefile}, поделенный на несколько
частей:

\lstinputlisting[style=mk,title=Makefile.mk,
linerange={1-1}]{cross/Makefile.mk}

\secdown

\secru{Выбор приложения}

\term{Приложение}\ --- программно-аппаратный комплекс, решающий определенную
задачу. В нашем случае приложение это \linux-часы, соответственно назовем
приложение:

\lstinputlisting[style=mk,title=Makefile.mk,
firstnumber=3,linerange={3-4}]{cross/Makefile.mk}

В переменной \verb|APP|\ указывается имя приложения. 
В зависимости от приложения, в главный Makefile нужно подключить кусок
соответствующего Makefile, выполняющего какие-то действия или настройки.

\lstinputlisting[style=mk,title=Makefile.mk,
firstnumber=12,linerange={12-12}]{cross/Makefile.mk}

\lstinputlisting[style=mk,title=app.clock.mk]{cross/app.clock.mk}


\secru{Аппаратная платформа}

Далее нам необходимо указать, на каком конкретном железе мы хотим запускать наше
приложение. При этом важно не только какой процессор на нем стоит, но важно
правильно сконфигурировать ядро \linux, чтобы были включены \emph{все
необходимые драйвера}, также в целевую сборку должен быть подключен правильный
скрипт инициализации оборудования и добавлены платформенно-специфичные
программы.

\lstinputlisting[style=mk]{cross/hw.mk}

С текущей версией книги в каталоге \file{cross}\ в комплекте идут наборы
скриптов сборки, поддерживающие следующие архитектуры:

\begin{itemize}
  \item \file{x86pc} Типовой ПК на интеловском процессоре, конфигурация
  прописана максимально совместимо, поэтому в результате сборка запустится на
  любом wintel компьютере, но не будут работать многие драйвера, потому что их
  нет. По необходимости вам нужно самостоятельно создать и прописать новую
  HW-конфигурацию для вашего железа, если захочется запустить звук, сеть и т.п.
  \item \file{qemu386}
  Конфигурация для запуска системы на эмуляторе QEMU в режиме i386, включены
  все необходимые драйвера, совместимые с эмулятором.
  \item \file{coubie1}
  Конфигурация для платы CoubieBoard версии 1, версии 2 у меня нет для
  тестирования.
  \item \file{rpi}
  Конфигурация для платы \raspi
\end{itemize}

\secru{Пакеты}

В обычных дистрибутивах пакетами называют архивы скомпилированных программ,
устанавливаемых в \linux-систему. В нашем случае кросс-компиляции, будем
называть \term{пакетом}\ архив исходных текстов определенной прграммы или
компонента системы, вместе с секциеями мейкфайла, выполняющего ее компиляцию.

Также пакет может быть не связан с компиляцией ПО, а выполнять какую-то
вспомогательную работу.

Пакеты запускаются по своему имени вручную с помощью команды
\begin{lstlisting}
~/book/cross$ make -f Makefile.mk <pacname1> [<packname2>...]
\end{lstlisting}

\secdown
\secru{Версии}

Для сборки кросс-компилятора нам потребуются следующие пакеты:

\begin{itemize}
  \item \file{binutils}\ ассемблер, линкер и утилиты для работы с
  \term{объектными файлами}
  \item \file{gmp}, \file{mpfr}, \file{mpc}\ библиотеки арифметики произвольной
  точности, необходимы для компиляции \file{gcc}
  \item \file{gcc}\ набор компиляторов GNU, нам нужны только Си и C++. 
\end{itemize}

В принципе можно поставить готовые кросс-компиляторы и библиотеки из репозитория
вашего дистрибутива, и не тратить время на сборку кросс-тулчейна. Но мы это
сделаем для иллюстрации возможности собрать свой тулчейн из исходников только
для одного конкретного проекта, не интегрируя собранный тулчен глобально в вашу
\linux-систему.

Такая сборка может понадобится в некоторых специфических случаях, как то желание
использовать самые свежие версии компиляторов, или изменить опции сборки
компилятора, отличающиеся от тех, которые заданы при сборке дистрибутивного.
Например в новой версии появилась расширенная поддержка вашего процессора,
подправили оптимизацию, или вам нужно наоборот собрать несколько пакетов старых
версий, которые не желают собираться новой версией gcc.

\lstinputlisting[style=mk,title=Makefile.mk,
firstnumber=14,linerange={14-18}]{cross/Makefile.mk}

Любая \emlinux-система включает в себя следующий обязательный набор базовых
пакетов:

\begin{enumerate}
  \item \file{linux} ядро ОС \linux
  \item \file{libc} стандартная библиотека Си, будем использовать специальную
  обрезанную версию \file{uClibc}, специально заточенную под применение во
  всраиваемых системах
  \item \file{bb} пакет \file{BusyBox}, набор множества стандартных утилит,
  тоже с обрезанными функциями для экономии ресурсов 
\end{enumerate}

\lstinputlisting[style=mk,title=Makefile.mk,
firstnumber=19,linerange={19-21}]{cross/Makefile.mk}

\secru{Полные имена пакетов}

включают номер версии пакета:

\lstinputlisting[style=mk,title=Makefile.mk,
firstnumber=23,linerange={23-30}]{cross/Makefile.mk}


\secup



\secru{Директории}

в которых будет производиться сборка, и формироваться целевая файловая система.

\secdown

\secrel{PWD}

Встроенная переменная, возвращает текущий каталог, из которого быз запущен
\file{make}, т.е. \file{/home/user/book/cross}.

\secrel{GZ}

Каталог для хранения архивов исходных текстов пакетов, закачанных из \internet\
(локальное зеркало). Если вы работаете сразу с несколькими проектами, создайте
отдельный каталог, например \file{/home/user/gz}, и из каждого проекта сделайте
на него симлинк \file{ln -s /home/user/gz gz}. В этом случае все архивы окажутся
общими для всех проектов, причем там могут лежать несколько версий одного
пакета.

\lstinputlisting[style=mk,title=Makefile.mk,
firstnumber=32,linerange={32-32}]{cross/Makefile.mk}

\secrel{SRC}

Сюда будут распаковываться исходные тексты пакетов

\lstinputlisting[style=mk,title=Makefile.mk,
firstnumber=33,linerange={33-33}]{cross/Makefile.mk}

\secrel{TMP}

Здесь будут создаваться и удаляться временные файлы и каталоги при сборке
пакетов\ --- out of tree сборка, за исключением пакетов, которые могут быть
собраны только в каталоге с исходными текстами.

\lstinputlisting[style=mk,title=Makefile.mk,
firstnumber=34,linerange={34-34}]{cross/Makefile.mk}

\secrel{TC}

От ToolChain, или TargetCompiler\ --- сюда поместим собранный кросс-тулчейн,
чтобы не засорять основную \linux-систему, и иметь возможность пользоваться
разными версиями компиляторов для разных проектов

\lstinputlisting[style=mk,title=Makefile.mk,
firstnumber=35,linerange={35-35}]{cross/Makefile.mk}

\secrel{ROOT}

В этом каталоге будем собирать \term{rootfs}\ --- корневую файловую систему

\lstinputlisting[style=mk,title=Makefile.mk,
firstnumber=36,linerange={36-36}]{cross/Makefile.mk}

\secrel{BOOT}

В этот подкаталог поместим откомпилированное ядро, файлы загрузчика и другие
файлы, обеспечивающие запуск системы до момента монтирования rootfs.

\lstinputlisting[style=mk,title=Makefile.mk,
firstnumber=37,linerange={37-37}]{cross/Makefile.mk}

\secup

\secru{Макросы команд}

\lstinputlisting[style=mk,title=Makefile.mk,
firstnumber=39,linerange={39-42}]{cross/Makefile.mk}

\secru{\file{dirs}}

Создание дерева каталогов для сборки.
Каталоги не зачищаемые пакетом \file{distclean}, прописаны отдельно. 

\lstinputlisting[style=mk,title=Makefile.mk,
firstnumber=44,linerange={44-47}]{cross/Makefile.mk}

\secru{\file{distclean}}

Зачистка проекта, применяется при полной пересборке

\lstinputlisting[style=mk,title=Makefile.mk,
firstnumber=49,linerange={49-52}]{cross/Makefile.mk}

\secru{Загрузка архивов исходников\ --- \file{gz}}

Длительная и потребляющая трафик операция, нужен онлайн.
По-хорошему архив исходников тут было бы желательно загружать через пиринговые
сети, а не нагружать зеркала. Если эта книга вдруг окажется популярной, нужно
будет сделать хранение архивов на торрентах.

\lstinputlisting[style=mk,title=Makefile.mk,
firstnumber=54,linerange={54-63}]{cross/Makefile.mk}

\secru{Шаблоны распаковки исходников}

Распаковка исходных текстов пакетов сделана через шаблоны:

\lstinputlisting[style=mk,title=Makefile.mk,
firstnumber=65,linerange={65-70}]{cross/Makefile.mk}

Для каждого типа архива задано правило, по которому архив должен быть
распакован, \file{make}\ контролирует факт распаковки по наличию файла
\file{README}, и отслеживает дату модификации файла архива.

\secru{Конфигурирование пакета}

Конфигурирование почти всех пакетов на этапе сборки выполняется с помощью
скрипта \file{src/package-version/configure}, созданного разработчиком пакета с
использованием средств \term{autotools}/\term{automake}.

Этот скрипт при своем запуске выполняет общую диагностику BUILD-системы, на
которой собирается пакет, учитывая значения параметров, заданные в командной
строке при его запуске.

В результате в текущем каталоге, где был запущен \file{configure}, создается
набор файлов, обеспечивающих дальнейшую сборку пакета через \file{make}\ и
установку его готовых бинарных файлов через \file{make install}.

Если сборка пакета выполняется в отдельном временном каталоге типа
\file{tmp/package-version}, такая сборка называется \term{out-of-tree}\ сборка.

Некоторые пакеты требуют сборку \textbf{в каталоге с исходниками}
\file{src/package}, и сборка называется \term{in-tree}\ сборка.

Общая часть вызова команды \file{configure}\ задана в переменной 

\lstinputlisting[style=mk,title=Makefile.mk,
firstnumber=72,linerange={72-72}]{cross/Makefile.mk}

\begin{itemize}
  \item \file{--diable-nls}\ отключает поддежку локализации (русского языка) в
  диагностическом выводе команд пакета
  
  Использование этого параметра немного ускоряет и упрощает процесс компиляции,
  и кроме того в IT\footnote{\ информационные технологии}-сфере разработчикам
  удобнее работать с диагностическим выводом компиляторов на английском, т.к.
  исчезают проблемы с кодировками между разными системами и выводом
  <<кракозябр>>. Также проще искать в \internet е подробную информацию об
  ошибках в программе.
  
  \emph{Не стоит забывать, что знание базового технического английского на
  уровне свободного чтения документации и диагностики\ --- один из минимально
  необходимых навыков IT-специалиста, наравне с умением пользоваться поисковыми
  системами типа Google. Если вы не можете это делать хотя бы со словарем,
  значит вы всего-лишь профнепрегодны для этой профессии.}
\end{itemize}

\secup
\secru{Сборка кросс-компилятора}
\secdown

При сборке \term{кросс-компилятора}\ используются две или три переменные,
содержащие \term{триплеты}\ указывающие на архитектуру каждой системы:

\begin{itemize}
  \item \file{BUILD}\ система, на которой производится сборка кросс-компилятора,
  например \file{i686-pc-linux-gnu} 
  \item \file{HOST}\ система, на котрой кросс-компилятор будет выполняться,
  например \file{i686-mingw32} (Windows без Cygwin-прослойки)
  \item \file{TARGET}\ целевая система, для которой компилируется ваш код,
  например \file{arm-none-eabi} (Raspberry Pi)
\end{itemize}

Соответствено для конфигурирования пакетов, которые будут работать либо на
билд-системе, либо компилироваться для целевой системы, нужно использовать
разные варианты запуска \file{configure}:

\lstinputlisting[style=mk,title=Makefile.mk,
firstnumber=74,linerange={74-75}]{cross/Makefile.mk}

\begin{itemize}
  \item параметр \file{--prefix}\ задает каталог, в который будет помещен
  откомпилированный пакет
\end{itemize}

\secrel{binutils}

\lstinputlisting[style=mk,title=Makefile.mk,
firstnumber=77,linerange={77-83}]{cross/Makefile.mk}

Удаляется если существует, и создается новый каталог out-of-tree сборки
\file{tmp/binutils-version}, и в нем запускается скрипт \file{configure}
в варианте для билд-системы, дополненный параметрами для \file{binutils}:

\begin{itemize}
  \item \file{--target}\ указываем триплет для целевой системы, binutils будут
  собираться для ее системы команд, набора регистров, параметров оптимизации
  \item в \file{CFG\_CPU}\ заданы параметры конкретного целевого процессора, см.
   файлы \file{hw/$<$system$>$.mk}, \file{cpu/$<$cpu$>$.mk}. Эти параметры будут
   жестко прописаны в настройки оптимизации и код кодогенерации binutils, и вы сможете
   скомпилировать вашу программу только конкретно для этого процессора, или
   полностью совместимого с ним на уровне машинного кода.
\end{itemize}

После отработки \file{configure}\ сразу запуститься сборка пакета (\file{make}),
и если не будет ошибок (это контролирует разделитель \verb|&&|), скрипты
инсталляции.

В результате в (ранее пустом) каталоге \file{\$(TC)=/home/user/book/cross/tc}\ 
появится дерево файлов откомпилированного \file{binutils}:

\begin{itemize}
  \item \file{bin/\$(TARGET)-as} \term{ассемблер}\ для целевой платформы
  \item \file{bin/\$(TARGET)-ld} \term{линкер}
  \item \file{bin/\$(TARGET)-objcopy} \term{копировщик секций} объектного кода
  (\file{.o}-файлы в формате ELF)
  \item \file{bin/\$(TARGET)-objdump} \term{дампер}\ позволяющий вывести как
  шестнадцатеричный дамп, дизассемблированный код для секций файлов
  .o (ELF). \emph{Эта же пограмма используется для получения бинарных и
  \file{.hex}\ файлов прошивок для микроконтроллеров}
  \item \file{bin/\$(TARGET)-size} программа выводит размеры секций .o
  \item \file{bin/\$(TARGET)-strip} \term{стриппер}\ удаляет отладочную
  информацию из объектных и исполняемых файлов\footnote{\ для \linux\ отличий в
  формате нет}
  \item \file{share/} документация в форматах man и info
\end{itemize}

На этом этапе вы уже можете пользоваться \file{кросс-ассемлером}\ для целевой
системы, компилируя их в файлы \term{объектного кода}, c помощью линкера
соединять несколько объектных файлов и файлов сторонних библиотек в один
исполняемый, уже содержащий полностью готовую к исполнению программу в машинном
коде целевой системы.

\secru{Сборка кросс-асемблера для AVR8 (Arduino/\ldots)}

Если вы пользуетесь этим руководством, с большой долей вероятности вам
потребуется подключать к вашей встраиваемой системе внешние контроллеры
ввода/вывода на базе микроконтроллеров Atmel AVR8 или Cortex-M.

В этом случае можно выполнить пару команд, собрав кросс-ассемблер для ходовых
микроконтроллеров, выполняющийся на host-системе:

\begin{lstlisting}[title=binutils-avr8]
user@builder:/home/user/cross$> make HW=avr8 CPU=atmega8 binutils
\end{lstlisting}

\emph{В этом примере показано, как вручную жестко переопределить значения
переменных в Makefile, изменяя его логику работы. В частности, мы собираем
\file{binutils}\ для другой платормы, причем переопределяем еще и тип
процессора.}

\secru{Сборка кросс-\file{binutils}\ для \cm{x}}

Аналогично можно собрать пакет кросс-\file{binutils}\ для микроконтроллеров
STM32 \cm{0\ldots 4}:

\begin{lstlisting}[title=\cm{x}]
user@builder$> make HW=STM32VLDISCOVERY binutils
\end{lstlisting}

\lstinputlisting[style=mk,title=cpu/arm/STM32F103RB.mk]{cross/cpu/arm/STM32F103RB.mk}
\lstinputlisting[style=mk,title=cpu/arm/CortexM3.mk]{cross/cpu/arm/CortexM3.mk}

\secup
