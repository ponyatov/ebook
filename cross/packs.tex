\secru{Пакеты}

В обычных дистрибутивах пакетами называют архивы скомпилированных программ,
устанавливаемых в \linux-систему. В нашем случае кросс-компиляции, будем
называть \term{пакетом}\ архив исходных текстов определенной прграммы или
компонента системы, вместе с секциеями мейкфайла, выполняющего ее компиляцию.

Также пакет может быть не связан с компиляцией ПО, а выполнять какую-то
вспомогательную работу.

Пакеты запускаются по своему имени вручную с помощью команды
\begin{lstlisting}
~/book/cross$ make -f Makefile.mk <pacname1> [<packname2>...]
\end{lstlisting}

\secdown
\secru{Версии}

Для сборки кросс-компилятора нам потребуются следующие пакеты:

\begin{itemize}
  \item \file{binutils}\ ассемблер, линкер и утилиты для работы с
  \term{объектными файлами}
  \item \file{gmp}, \file{mpfr}, \file{mpc}\ библиотеки арифметики произвольной
  точности, необходимы для компиляции \file{gcc}
  \item \file{gcc}\ набор компиляторов GNU, нам нужны только Си и C++. 
\end{itemize}

В принципе можно поставить готовые кросс-компиляторы и библиотеки из репозитория
вашего дистрибутива, и не тратить время на сборку кросс-тулчейна. Но мы это
сделаем для иллюстрации возможности собрать свой тулчейн из исходников только
для одного конкретного проекта, не интегрируя собранный тулчен глобально в вашу
\linux-систему.

Такая сборка может понадобится в некоторых специфических случаях, как то желание
использовать самые свежие версии компиляторов, или изменить опции сборки
компилятора, отличающиеся от тех, которые заданы при сборке дистрибутивного.
Например в новой версии появилась расширенная поддержка вашего процессора,
подправили оптимизацию, или вам нужно наоборот собрать несколько пакетов старых
версий, которые не желают собираться новой версией gcc.

\lstinputlisting[style=mk,title=Makefile.mk,
firstnumber=14,linerange={14-18}]{cross/Makefile.mk}

Любая \emlinux-система включает в себя следующий обязательный набор базовых
пакетов:

\begin{enumerate}
  \item \file{linux} ядро ОС \linux
  \item \file{libc} стандартная библиотека Си, будем использовать специальную
  обрезанную версию \file{uClibc}, специально заточенную под применение во
  всраиваемых системах
  \item \file{bb} пакет \file{BusyBox}, набор множества стандартных утилит,
  тоже с обрезанными функциями для экономии ресурсов 
\end{enumerate}

\lstinputlisting[style=mk,title=Makefile.mk,
firstnumber=19,linerange={19-21}]{cross/Makefile.mk}

\secru{Полные имена пакетов}

включают номер версии пакета:

\lstinputlisting[style=mk,title=Makefile.mk,
firstnumber=23,linerange={23-30}]{cross/Makefile.mk}


\secup

